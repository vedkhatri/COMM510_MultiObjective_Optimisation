\documentclass[conference,a4paper]{IEEEtran}
\IEEEoverridecommandlockouts
\usepackage{cite}
\usepackage{amsmath,amssymb,amsfonts}
\usepackage{algorithmic}
\usepackage{graphicx}
\usepackage{textcomp}
\usepackage{xcolor}
% \usepackage{flushend} % enable and check as part of the final
                        % checking / proof reading
\usepackage[detect-all]{siunitx}
\usepackage{hyperref}

\def\BibTeX{{\rm B\kern-.05em{\sc i\kern-.025em b}\kern-.08em
    T\kern-.1667em\lower.7ex\hbox{E}\kern-.125emX}}
\begin{document}

\title{Effects of noise on multi-objective optimiser performance}

\author{\IEEEauthorblockN{Dushyant Khatri}       
\IEEEauthorblockA{\textit{Department of Computer Science}\\
\textit{University of Exeter, Exeter, UK}
}
}

\maketitle

\begin{abstract}
  Noise on the objective functions has been seen to be detrimental to the performance of standard evolutionary multi-objective optimisers. This has led to the development of bespoke optimisers for this type of problem. In addition to these, it has been observed that small amounts of noise can be beneficial to some multi-objective search algorithms, even using optimisers that assume no noise is present. This project investigates the combination of problem type, noise type and magnitude, and explores the underlying relationship between performance of standard optimisers in noisy problems are, and the situation led to improved performance when noise is present. The paper concludes with comparisons drawn and interpreting results in form of comparison plots of different combinations.
\end{abstract}


\section{Introduction}\label{sec:introduction}
Noise is an unavoidable and an inevitable element that is present in many real-world optimisation problems \cite{branke2003}. When modelling these problems, noise causes variation to the output, leading to a solution that may not reflect the true coefficients. A common group of multi-objective algorithms used in tackling noisy 

\subsection{Research objective and scope}\label{sec: oneA}
In this project, the effects of noise on the performance of multi-objective optimisation algorithms (i.e., optimisers) are investigated. The scope of this investigation is restricted to Multi-objective Evolutionary Algorithms (MOEAs). The reason to why EAs are considered is discussed further in section II.A. 

\section{Background}\label{sec: two}
Our focus in this study is restricted to Multi-objective Optimisation Problems (MOPs). 
The MOPs of interest for this study are restricted to EAs: Non-dominated Sorting Genetic Algorithm - II (NSGA-II), Strength Pareto Evolutionary Algorithm 2 (SPEA2), Pareto Archived Evolution Strategy (PAES), and Indicator-based Evolutionary Algorithm ($IBEA_{\epsilon+}$). These algorithms are briefly described in the next section.

\subsection{Multi-objective Optimisation and Evolutionary Algorithms}\label{sec:twoA}

\subsection{Noise}\label{sec:twoB}

\subsection{Related Work}\label{sec: twoC}

\section{Methodology}\label{sec: three}

\subsection{Simulation of Noise}\label{sec: threeA}

\subsection{Tools and Techniques}\label{sec: threeB}

\section{Analysis}\label{sec: four}

\section{Results}\label{sec: five}

\section{Conclusion}\label{sec: six}




\section{example figure and table}
\begin{table}[htbp]
\caption{Table Type Styles}
\begin{center}
\begin{tabular}{|c|c|c|c|}
\hline
\textbf{Table}&\multicolumn{3}{|c|}{\textbf{Table Column Head}} \\
\cline{2-4} 
\textbf{Head} & \textbf{\textit{Table column subhead}}& \textbf{\textit{Subhead}}& \textbf{\textit{Subhead}} \\
\hline
copy& More table copy$^{\mathrm{a}}$& &  \\
\hline
\multicolumn{4}{l}{$^{\mathrm{a}}$Sample of a Table footnote.}
\end{tabular}
\label{tab1}
\end{center}
\end{table}

\begin{figure}[htbp]
\centerline{\includegraphics[width=0.95\linewidth]{figure1.png}}
\caption{Example of a figure caption.}
\label{fig}
\end{figure}


\bibliographystyle{IEEEtran}
\bibliography{references.bib}


\end{document}
